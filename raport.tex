\documentclass[12pt,a4paper,twoside,openright,fleqn]{mwrep}
\usepackage[latin2]{inputenc}
\usepackage[OT4,plmath]{polski}
\usepackage{amsmath}
\usepackage{amssymb}
\usepackage{graphicx}
%\usepackage[a4paper,left=3.5cm,right=2.5cm,top=2.5cm,bottom=2.5cm]{geometry}
\usepackage[width=16cm,height=25cm]{geometry}
\usepackage{fancyvrb}
\usepackage{listings}
%\usepackage{bold-extra}
\usepackage[unicode=true]{hyperref}
\usepackage{color}
\usepackage{cite}

\pagestyle{uheadings}

% Definicje ustalaj�ce wygl�d wydruk�w program�w
\definecolor{darkgray}{rgb}{0.95,0.95,0.95}
\lstset{frame=tb, backgroundcolor=\color{darkgray}}
\lstset{numbers=left, numberstyle=\tiny, stepnumber=2, numbersep=5pt}
\lstset{basicstyle=\ttfamily, keywordstyle=\color{blue}\bfseries}


\hypersetup{
    pdfpagemode=UseOutlines,   % otwiera dokument w trybie jednej strony
    pdfpagelayout=SinglePage,  %
    pdfstartpage=1,            % na podanej stronie
    bookmarksopen=true,        % rozwini�cie zak�adek
    bookmarksopenlevel=1,   % do jakiego poziomu
    colorlinks=true,   % kolorowanie odno�nik�w zamiast ramki wok� nich
    citecolor=cyan,   % kolor odno�nik�w do bibliografii, domy�lnie zielony
    filecolor=red,   % kolor odno�nik�w do lokalnych plik�w, domy�nie magenta
    linkcolor=blue,   % kolor odno�nik�w wewn�trznych, domy�lnie czerwony
    menucolor=green,   % kolor pozycji menu Acrobata, domy�lnie czerwony
    urlcolor=blue,   % kolor odno�nik�w do adres�w internetowych, domy�lnie cyan
                               % DANE DOKUMENTACJI
    pdfauthor={ARR2016},
    pdftitle={Projekt przej�ciowy --- Symulator labolatorium L1.5},
 }

                                   % DEFINICJE W�ASNE
\newcommand{\dd}{\text{d}}
\newtheorem{uwaga}{Uwaga}
\newtheorem{twr}{Twierdzenie}


\title{Projekt przej�ciowy --- Symulator labolatorium L1.5}
\author{ARR 2016}
\date{\today}


\begin{document}
\VerbatimFootnotes

\maketitle
\tableofcontents

\chapter{Wst�p}

\section{Motywacja}

Projekt wirtualnego laboratorium powsta� jako pomoc dydaktyczna do zaj�� laboratoryjnych wykonywanych na robotach Pioneer w �rodowisku ROS realizowanych na Politechnice Wroc�awskiej. Dzi�ki �rodowisku studenci mog� w ramach przygotowania do zaj�� wykona� �wiczenia na robotach Pioneer w systemie ROS dost�pnych w wirtualnym laboratorium. Ponadto, �rodowisko pozwala zainteresowanym na zapoznanie si� ze sposobem budowy �rodowiska jako odizolowanego kontenera Dockera i jego obs�ug�.
\section{Cel projektu}

Celem projektu by�o stworzenie wirtualnego laboratorium umo�liwiaj�cego obs�ug� robot�w pionier w systemie ROS w �rodowisku odpowiadaj�cym rzeczywistemu laboratorium.

Dostarczone �rodowisko mia�o umo�liwi� wyb�r laboratorium i robot�w pionier oraz pozwala� na dodawanie element�w sceny takich jak przeszkody. Z wykorzystaniem platformy ROS zosta�o udost�pnione sterowanie robotami. Symulator pozwala na rejestracj� i zapis pozycji robota w postaci wykres�w oraz danych.

�rodowisko dostarczone zosta�o w postaci kontenera dockera, na kt�rym znajduje si� system operacyjny ubuntu 16.04 ROS Kinetic oraz gazebo 7.5. 

Osoba, kt�rej zadanie ogranicza� si� b�dzie do wykonania �wicze� laboratoryjnych, powinna zapozna� si� z instrukcj� z dodatku A. W celu u�atwienia rozwoju i  modyfikacji projektu stworzony zosta� dodatek B.


\bibliographystyle{plabbrv}
\bibliography{bibliografia}

\end{document}
