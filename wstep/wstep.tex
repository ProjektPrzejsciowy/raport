\chapter{Wst�p}
�rodowisko do symulowania dzia�ania robot�w w warunkach laboratorium L1.5 zosta� wykonany na zaj�ciach projektu przej�ciowego grupy ARR. �rodowisko jest dostarczony w postaci kontenera dockera na kt�rym znajduje si� system operacyjny ubuntu 16.04 ros kinetic oraz gazebo 7.5. 
\begin{equation}
x
\end{equation}

Celem projektu by�o umo�liwienie przygotowania si� studentom do zaj�� laboratoryjnych poprzez testowanie napisanych przez siebie program�w w �rodowisku zbli�onym do udost�pnianej przestrzeni podczas zaj�� praktycznych. Dostarczony projekt umo�liwia wyb�r laboratorium i robot�w pionier oraz pozwala na dodawanie element�w sceny takich jak przeszkody. Z wykorzystaniem �rodowiska ros mo�na sterowa� robotami. Symulator pozwala r�wnie� na zmian� pozycji robota. % oraz rejestracj� i zapis jego pozycji

\begin{equation}
y
\end{equation}
Osoba kt�rej zadanie ogranicza� si� b�dzie do wykonania �wicze� laboratoryjnych powinna zapozna� si� z instrukcj� z dodatku A. W celu u�atwienia rozwoju i  modyfikacji projektu stworzony zosta� dodatek B.

\section{Motywacja}

Projekt wirtualnego laboratorium powsta� jako pomoc dydaktyczna do zaj�� laboratoryjnych wykonywanych na robotach Pioneer w �rodowisku ROS realizowanych na Politechnice Wroc�awskiej. Dzi�ki �rodowisku studenci mog� w ramach przygotowania do zaj�� wykona� �wiczenia na robotach Pioneer w systemie ROS dost�pnych w wirtualnym laboratorium. Ponadto, �rodowisko pozwala zainteresowanym na zapoznanie si� ze sposobem budowy �rodowiska jako odizolowanego kontenera Dockera i jego obs�ug�.
\section{Cel projektu}

Celem projektu by�o stworzenie wirtualnego laboratorium umo�liwiaj�cego obs�ug� robot�w pionier w systemie ROS w �rodowisku odpowiadaj�cym rzeczywistemu laboratorium.

Dostarczone �rodowisko mia�o umo�liwi� wyb�r laboratorium i robot�w pionier oraz pozwala� na dodawanie element�w sceny takich jak przeszkody. Z wykorzystaniem platformy ROS zosta�o udost�pnione sterowanie robotami. Symulator pozwala na rejestracj� i zapis pozycji robota w postaci wykres�w oraz danych.

�rodowisko dostarczone zosta�o w postaci kontenera dockera, na kt�rym znajduje si� system operacyjny ubuntu 16.04 ROS Kinetic oraz gazebo 7.5. 

Osoba, kt�rej zadanie ogranicza� si� b�dzie do wykonania �wicze� laboratoryjnych, powinna zapozna� si� z instrukcj� z dodatku A. W celu u�atwienia rozwoju i  modyfikacji projektu stworzony zosta� dodatek B.
