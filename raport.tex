\documentclass[12pt,a4paper,twoside,openright,fleqn]{mwrep}
%\documentclass[11pt, a4paper]{mwrep}
\usepackage[latin2]{inputenc}
\usepackage[OT4,plmath]{polski}
\usepackage{amsmath}
\usepackage{amssymb}
\usepackage{graphicx}
%\usepackage[a4paper,left=3.5cm,right=2.5cm,top=2.5cm,bottom=2.5cm]{geometry}
\usepackage[width=16cm,height=25cm]{geometry}
\usepackage{fancyvrb}
\usepackage{listings}
%\usepackage{bold-extra}
\usepackage[unicode=true]{hyperref}
\usepackage{color}
\usepackage{cite}
\usepackage{float}

\pagestyle{uheadings}

% Definicje ustalaj±ce wygl±d wydruków programów
\definecolor{darkgray}{rgb}{0.95,0.95,0.95}
\lstset{frame=tb, backgroundcolor=\color{darkgray}}
\lstset{numbers=left, numberstyle=\tiny, stepnumber=2, numbersep=5pt}
\lstset{basicstyle=\ttfamily\footnotesize, keywordstyle=\color{blue}\bfseries}
\lstset{tabsize=2}


\hypersetup{
    pdfpagemode=UseOutlines,   % otwiera dokument w trybie jednej strony
    pdfpagelayout=SinglePage,  %
    pdfstartpage=1,            % na podanej stronie
    bookmarksopen=true,        % rozwiniêcie zak³adek
    bookmarksopenlevel=1,   % do jakiego poziomu
    colorlinks=true,   % kolorowanie odno¶ników zamiast ramki wokó³ nich
    citecolor=cyan,   % kolor odno¶ników do bibliografii, domy¶lnie zielony
    filecolor=red,   % kolor odno¶ników do lokalnych plików, domy¶nie magenta
    linkcolor=blue,   % kolor odno¶ników wewnêtrznych, domy¶lnie czerwony
    menucolor=green,   % kolor pozycji menu Acrobata, domy¶lnie czerwony
    urlcolor=blue,   % kolor odno¶ników do adresów internetowych, domy¶lnie cyan
                               % DANE DOKUMENTACJI
    pdfauthor={ARR2016},
    pdftitle={Projekt Przej�ciowy --- Symulator laboratorium L1.5},
 }
                                   % DEFINICJE W£ASNE
\newcommand{\dd}{\text{d}}
\newtheorem{uwaga}{Uwaga}
\newtheorem{twr}{Twierdzenie}

\title{Projekt Przej�ciowy --- Symulator laboratorium L1.5}
\author{ARR 2016}
\date{\today}


\begin{document}
\VerbatimFootnotes
\newcommand{\MONTH}{%
  \ifcase\the\month
  \or stycze� %January% 1
  \or luty %February% 2
  \or marzec %March% 3
  \or kwiecie� %April% 4
  \or maj %May% 5
  \or czerwiec %June% 6
  \or lipiec %July% 7
  \or sierpie� %August% 8
  \or wrzesie� %September% 9
  \or pa�dziernik %October% 10
  \or listopad %November% 11
  \or grudzie� %December % 12
  \fi}

\newgeometry{left=1.5cm,right=1.5cm,bottom=1.5cm,top=1.5cm} %wymaga geometry 5.0+
\thispagestyle{empty}
%\vspace*{-2.5cm}
\noindent
\fbox{\parbox{\linewidth}{
\vspace{5mm}
\makebox[\linewidth][l]{
\includegraphics[scale=1]{title/figures/znak-pwr_poziom-pl-cmyk.eps}

}\\
\rule{\linewidth}{1.5pt}
\mbox{
\begin{minipage}[b]{\linewidth}
\begin{center}
\vspace{25mm}
\textbf{\LARGE Projekt Przej�ciowy}\\
\textbf{\large Wirtualne Laboratorium L1.5}
\end{center}
\vspace{2.5cm}
\begin{minipage}[t]{0.45\textwidth}
\end{minipage}


\begin{flushright}
\parbox{6cm}{
Prowadz�cy:\\
dr in�. Robert Muszy�ski\\
dr in�. Janusz Jakubiak
}\end{flushright}
\hspace{5mm}
\parbox{5cm}{
\begin{tabbing}
Krzysztof Kwieci�ski  \\
Kamil Bogus  \\
Micha� Burdka  \\
Krzysztof Kuczy�ski  \\
Witold Lipieta  \\
Micha� Pr�dkiewicz  \\
Artur B�a�ejewski  \\
Pawe� Jachimowski  \\
Rafa� Cymi�ski  \\
J�drzej Stolarz  \\

\end{tabbing}

Rok akademicki 2016/17\\
}\\[4cm]
\begin{center}
Wroc�aw, \MONTH \the\year
\end{center}
\vspace{1cm}
\end{minipage}
}}}
\restoregeometry
\newpage

\tableofcontents

\chapter{Wst�p}
�rodowisko do symulowania dzia�ania robot�w w warunkach laboratorium L1.5 zosta� wykonany na zaj�ciach projektu przej�ciowego grupy ARR. �rodowisko jest dostarczony w postaci kontenera dockera na kt�rym znajduje si� system operacyjny ubuntu 16.04 ros kinetic oraz gazebo 7.5. 

Celem projektu by�o umo�liwienie przygotowania si� studentom do zaj�� laboratoryjnych poprzez testowanie napisanych przez siebie program�w w �rodowisku zbli�onym do udost�pnianej przestrzeni podczas zaj�� praktycznych. Dostarczony projekt umo�liwia wyb�r laboratorium i robot�w pionier oraz pozwala na dodawanie element�w sceny takich jak przeszkody. Z wykorzystaniem �rodowiska ros mo�na sterowa� robotami. Symulator pozwala r�wnie� na zmian� pozycji robota. % oraz rejestracj� i zapis jego pozycji

Osoba kt�rej zadanie ogranicza� si� b�dzie do wykonania �wicze� laboratoryjnych powinna zapozna� si� z instrukcj� z dodatku A. W celu u�atwienia rozwoju i  modyfikacji projektu stworzony zosta� dodatek B.

\section{Motywacja}

Projekt wirtualnego laboratorium powsta� jako pomoc dydaktyczna do zaj�� laboratoryjnych wykonywanych na robotach Pioneer w �rodowisku ROS realizowanych na Politechnice Wroc�awskiej. Dzi�ki �rodowisku studenci mog� w ramach przygotowania do zaj�� wykona� �wiczenia na robotach Pioneer w systemie ROS dost�pnych w wirtualnym laboratorium. Ponadto, �rodowisko pozwala zainteresowanym na zapoznanie si� ze sposobem budowy �rodowiska jako odizolowanego kontenera Dockera i jego obs�ug�.
\section{Cel projektu}

Celem projektu jest stworzenie wirtualnego laboratorium umo�liwiaj�cego obs�ug� robot�w Pioneer w systemie ROS w �rodowisku odpowiadaj�cym warunkom rzeczywistego laboratorium. Dostarczone �rodowisko umo�liwia wyb�r laboratorium i robot�w Pioneer oraz pozwala na dodawanie element�w sceny, takich jak przeszkody. Sterowanie robotami mo�liwe jest dzi�ki platformie ROS. Symulator pozwala na rejestracj� i zapis pozycji robota w postaci graficznej oraz numerycznej.

�rodowisko dostarczone jest w postaci kontenera Dockera, na kt�rym znajduje si� system operacyjny Ubuntu 16.04, ROS Kinetic oraz Gazebo 7.5. Osoba, kt�rej zadanie ogranicza� si� b�dzie do pracy z systemem (np. w celu wykonania �wicze� laboratoryjnych), powinna zapozna� si� z instrukcj� zawart� w dodatku A. Materia�y zamieszczone w dodatku B maj� u�atwi� rozw�j i modyfikacje projektu.

\chapter{Wykorzystane narz�dzia}
\section{Docker}

\begin{figure}[htbp]
 \centering
 \includegraphics[width=0.4\textwidth]{wykorzystane_narzedzia/materialy/docker.png}
 \label{fig:docker}
\end{figure}

Docker to narz�dzie szturmem zdobywaj�ce popularno�� na serwerach, szczeg�lnie w �rodowiskach chmurowych, gdzie z powodzeniem wspiera lub czasem nawet zast�puje klasyczn� wirtualizacj� oferowan� przez rozwi�zania typu VMware lub XEN.

Docker dzia�a tylko na j�drze Linux i pozwala uruchamia� tylko aplikacje przeznaczone dla Linuxa, ale dla wszystkich u�ytkownik�w Windows i Mac jest przygotowane narz�dzie Docker Toolbox, kt�re pozwala zainstalowa� Dockera w minimalnej maszynie wirtualnej pod kontrol� VirtualBox'a.

Zdecydowan� przewag� Dockera nad wirtualizacj� jest mo�liwo�� uruchomienia aplikacji w wydzielonym kontenerze, ale bez konieczno�ci emulowania ca�ej warstwy sprz�towej i~systemu operacyjnego. Docker uruchamia w kontenerze tylko i wy��cznie proces(y) aplikacji i nic wi�cej. Efektem jest wi�ksza efektywno�� wykorzystania zasob�w sprz�towych, co przy rozproszonych aplikacjach instalowanych do tej pory na kilkunastu b�d� kilkudziesi�ciu wirtualnych maszynach przynosi konkretne oszcz�dno�ci.

Docker zast�puje wirtualizacj� przez stosowanie konteneryzacji. Konteneryzacja polega na tym, �e umo�liwia uruchomienie wskazanych proces�w aplikacji w wydzielonych kontenerach, kt�re z punktu widzenia aplikacji s� odr�bnymi instancjami �rodowiska uruchomieniowego. Ka�dy kontener posiada wydzielony obszar pami�ci, odr�bny interfejs sieciowy z w�asnym prywatnym adresem IP oraz wydzielony obszar na dysku, na kt�rym znajduje si� zainstalowany obraz systemu operacyjnego i wszystkich zale�no�ci / bibliotek potrzebnych do dzia�ania aplikacji.

Kontenery Dockera dzia�aj� niezale�nie od siebie i do chwili, w kt�rej �wiadomie wska�emy zale�no�� pomi�dzy nimi, nic o sobie nie wiedz�.

Dwie g��wne korzy�ci p�yn�ce z korzystania z Dockera to �atwo�� tworzenia �rodowisk deweloperskich oraz uproszczenie proces�w dostarczania gotowych aplikacji na docelowe �rodowiska.

Zmor� ka�dego programisty jest tworzenie �rodowiska deweloperskiego na potrzeby ka�dego kolejnego projektu. Wiadomo, �e ka�dy projekt b�dzie dzia�a� na innej bazie danych, innym kontenerze aplikacji z inn� list� dodatkowych us�ug, kt�re do czasu pojawienia si� narz�dzi typu Vagrant instalowane by�y bezpo�rednio na laptopie programisty. Utrzymanie kilku �rodowisk dla wielu projekt�w bywa�o niemo�liwe. Vagrant w pewien spos�b rozwi�zuje ten problem przez tworzenie wirtualnego �rodowiska instalowanego za pomoc� odpowiednich skrypt�w, ale to rozwi�zanie nie rozwi�zuje wszystkich problem�w: nadal musimy napisa� skrypty instaluj�ce wszystkie zale�no�ci i potrzebny jest mocny sprz�t �eby ud�wign�� pe�ne �rodowisko w trybie wirtualizacji. Kilka takich �rodowisk na laptopie mo�e te� skutecznie zaj�� ca�� przestrze� dyskow�.

Z pomoc� przychodzi Docker, kt�ry pozwala zbudowa� �rodowisko deweloperskie bez wirtualizacji i bez wi�kszego wysi�ku zwi�zanego z instalacj� oprogramowania. Docker pozwala wykorzystywa� gotowe obrazy zainstalowanych system�w, aplikacji i baz danych, kt�re zosta�y wcze�niej przygotowane i umieszczone w publicznym rejestrze. Rejestr jest dost�pny za darmo i zawiera obrazy oficjalnie budowane przez opiekun�w / tw�rc�w konkretnych rozwi�za�: \url{https://hub.docker.com/explore/}. Dzi�ki temu, je�li chcemy u�ywa� V-Rep, Gazebo czy ROS, jest du�a szansa na to, �e gotowy obraz b�dziemy mogli pobra� z repozytorium i nie traci� czasu na jego przygotowanie. Je�li jednak nie znajdziemy tego czego szukamy, to zawsze mo�emy zbudowa� w�asny obraz bazuj�c na jednym z bardziej generycznych zawieraj�cych tylko zainstalowany system operacyjny (Ubuntu, Fedora, itp.) lub zainstalowane �rodowisko uruchomieniowe (Java, Python czy ASP.NET).

\subsection{Podstawowe elementy}
\textbf{Docker Engine} to g��wna sk�adowa i aplikacja typu klient-serwer, z�o�ona z trzech element�w: klienta (CLI), serwera (daemon-a) oraz REST API.
\begin{figure}[htbp]
 \centering
 \includegraphics[width=0.7\textwidth]{wykorzystane_narzedzia/materialy/docker_structure.png}
 \label{fig:docker_structure}
\end{figure}

\textbf{Daemon Docker-a} jest centralnym miejscem, z kt�rego nast�puje zarz�dzanie kontenerami jak i obrazami. Dotyczy to zar�wno ich pobierania, budowy czy uruchamiania. Polecenia do wykonywania tych czynno�ci s� przesy�ane przez klienta z wykorzystaniem Docker REST API. Same obrazy s� natomiast przechowywane w repozytorium obraz�w, czy to publicznym czy prywatnym np. \textbf{Docker Hub}. W tym miejscu nale�y zaznaczy�, �e poza wieloma oficjalnymi obrazami udost�pniane s� r�wnie� nieoficjalne: budowane przez spo�eczno�� Docker-a.

W odr�nieniu od maszyn wirtualnych, kontenery wymagaj� du�o mniejszych zasob�w do samego uruchomienia, a i sam czas ich uruchomienia jest znacz�co ni�szy. Zosta�o to jednak uzyskane kosztem zmniejszenia izolacji pomi�dzy kontenerem, a samym systemem operacyjnym : poprzez wsp�dzielenie j�dra systemu.

\textbf{Obrazy} s� tak naprawd� szablonami w trybie read-only, z kt�rych kontenery s� uruchamiane. Sk�adaj� si� one z wielu warstw (layer-�w), kt�re, dzi�ki zastosowaniu ujednoliconego systemu plik�w (UFS), Docker ��czy w jeden konkretny obraz. Podstaw� ka�dego jest obraz bazowy, np. Ubuntu, na kt�ry nak�adane s� kolejne warstwy. Ka�da kolejna czynno�� (instrukcja) wykonywana na obrazie bazowym tworzy kolejn� warstw�, np. wykonanie komendy czy utworzenie pliku/katalogu. Komplet instrukcji tworz�cych obraz jest przechowywany w pliku Dockerfile. Podczas ��dania pobrania obrazu przez klienta, plik ten jest przetwarzany, czego wynikiem jest finalny obraz.

Poprzez zastosowanie warstw uzyskano bardzo niskie zu�ycie przestrzeni dyskowej, poniewa� podczas zmiany obrazu czy jego aktualizacji, budowana jest nowa warstwa, kt�ra zast�puje poprzedni� (aktualizowan�). Pozosta�e warstwy pozostaj� nienaruszone. Oznacza to, i� mo�liwe jest wsp�dzielenie warstw tylko do odczytu pomi�dzy kontenerami, czego efektem jest du�o ni�sze zu�ycie przestrzeni dyskowej, w por�wnaniu do standardowych VM.

\textbf{Registry}, czyli repozytorium obraz�w, jest faktycznym miejscem przechowywania obraz�w. Mo�e by� publiczne b�d� prywatne, lokalne b�d� zdalne. Najpopularniejszym repozytorium jest Docker Hub, kt�ry oferuje wiele dodatkowych funkcjonalno�ci, np. mo�liwo�� utworzenia repozytorium prywatnego. Oczywi�cie istniej� inne platformy, kt�re pozwalaj� na przechowywanie swoich obraz�w, jak np.  Quay.io, ale mo�liwe jest tak�e utworzenie w�asnej biblioteki: czy to lokalnie na komputerze, na kt�rym zosta� zainstalowany Docker czy zdalnie, na jednej z innych maszyn, kt�rymi zarz�dzamy.

Wspomniany wielokrotnie \textbf{kontener} to efekt ujednolicenia warstw tylko do odczytu oraz pojedynczej warstwy do odczytu i zapisu, dzi�ki kt�rej mo�liwe jest funkcjonowanie wymaganych zada�. Kontener wykonuje okre�lone wcze�niej zadanie, najcz�ciej jedno. Zawiera system operacyjny, pliki u�ytkownika, a tak�e tzw. metadane, dodawane automatycznie podczas tworzenia b�d� startu kontenera.

Kontener okre�lany jest r�wnie� jako �rodowisko wykonywalne Dockera. Mo�e przyjmowa� jeden z pi�ciu stan�w:
\begin{itemize}
\item created : utworzony, gotowy do uruchomienia,
\item up : dzia�aj�cy, wykonuj�cy zadanie,
\item exited : wy��czony, w trybie bezczynno�ci po zako�czeniu zadania,
\item paused : wstrzymany,
\item restarting : w trakcie ponownego uruchamiania.
\end{itemize}

Czas dzia�ania kontenera jest zale�ny od zadania, kt�re wykonuje. Samo uruchomienie sk�ada si� z siedmiu krok�w:
\begin{enumerate}
\item Pobrania wybranego obrazu, pod warunkiem, �e nie zosta� ju� pobrany wcze�niej.
\item Utworzenia kontenera.
\item Za�adowania systemu plik�w i utworzenia warstwy do odczytu i zapisu.
\item Zainicjowania sieci b�d� mostka sieciowego.
\item Konfiguracji sieci (adresu IP).
\item Uruchomienia zadania.
\item Przechwytywania wyj�cia i prowadzenia dziennika zdarze�.
\end{enumerate}

Mo�liwe jest ��czenie (linkowanie) kontener�w, przez co zyskuj� one bezpo�rednie po��czenie ze sob�. Same zadania wykonywane przez kontener s� tak samo wydajne, jakby by�y uruchamianie bezpo�rednio w systemie gospodarza.
 \section{Gazebo}
  
 \begin{figure}[htbp]
  \centering
  \includegraphics[width=0.4\textwidth]{wykorzystane_narzedzia/materialy/gazebo.jpg}
  \label{fig:blender}
 \end{figure}
 
  Symulator robotów jest doskonałym narzędziem dla każdej osoby zajmującej się robotyką. Pozwala szybko przetestować różne algorytmy i konstrukcje oraz skomplikowane systemy realizujące niecodzienne scenariusze. Jednym z takich narzędzi jest darmowy program Gazebo, przeznaczony do tworzenia dokładnych i efektywnych symulacji 3D robotów działających w złożonych środowiskach. Posiada zaawansowany silnik fizyki, wysokiej jakości grafikę oraz wygodne i programowalne interfejsy. Symulator dostępny jest dla systemu Linux na licencji Apache 2.0.
  Program Gazebo powstał na Uniwersytecie Południowej Kalifornii jako część systemu The Player Project, w którego skład wchodził również symulator 2D - Stage. W 2011 roku symulator ten został zintegrowany z bibliotekami Robot Operating System(ROS), co przyczyniło się do znacznego zwiększenia jego popularności. Rozwojem Gazebo zajmuje się teraz Open Source Robotic Foundation. Silnik graficzny wykorzystany w Gazebo to OGRE (Object Oriented Graphics Rendering Engine). Odpowiada on za wygląd użytych modeli, oświetlenie oraz cienie. Poprawnie odwzorowaną fizykę (między innymi model zderzeń i kolizji) zapewnia silnik ODE (Open Dynamics Engine). Gazebo daje możliwość symulowania takich sensorów jak: czujniki odległości, kamery, skanery 3D czy moduły GPS. Program pozwala na dodanie do każdego modelu kontrolera, sterującego jego pracą.Możliwe jest tworzenie prostych modeli bezpośrednio w Gazebo oraz importowanie bardziej złożonych, zaprojektowanych w zewnętrznych programach do grafiki 3D.
  
Podczas wyboru symulatora najistotniejsza była możliwość integracji z ROSem, a pod tym względem Gazebo jest bezkonkurencyjne. Nie bez znaczenia jest również dobra dokumentacja dostępna na stronie http://gazebosim.org/.

\section{ROS}
%http://www.ros.org/about-ros/
%http://wiki.ros.org/ROS/Introduction
\textsf{ROS} jest meta-systemem operacyjnym rozwijanym w ramach ruchu wolnego oprogramowania. Zawiera: podstawowe procesy systemowe obs�uguj�ce urz�dzenia sprz�towe robota, sterowanie niskopoziomowe, implementacje wykonywania typowych funkcji, komunikacj� mi�dzyw�tkow� oraz zarz�dzanie pakietami. Opr�cz tego dostarcza u�ytkownikowi narz�dzia i~biblioteki pozwalaj�ce na tworzenie i~uruchomianie programu r�wnocze�nie na wielu komputerach. Jest platform� programistyczn� przeznaczon� do tworzenia oprogramowania steruj�cego robotami.
\textsf{ROS} jest szczeg�owo opisany na stronie internetowej projektu~\cite{website:gazebo}.  

\subsection{Struktura}
%http://rab.ict.pwr.wroc.pl/dydaktyka/instrukcje/ros_instrukcja.pdf
%http://wiki.ros.org/ROS/Concepts
Dzi�ki swojej strukturze (zbi�r narz�dzi, bibliotek i~konwencji) \textsf{ROS} znacz�co upraszcza proces modelowania z�o�onych zachowa� robota jednocze�nie zapewniaj�c �atwo�� przeno�no�ci kodu pomi�dzy r�nymi platformami robotycznymi. Zasada dzia�ania projektu w~\textsf{ROS} polega na komunikacji \textit{peer-to-peer} lu�no po��czonych ze sob� proces�w (mog� by� uruchomione na r�nych maszynach) za pomoc� infrastruktury komunikacyjnej zapewnianej przez platform�. Idea struktury �rodowiska przedstawiona jest na rysunku~\ref{fig:ros_struktura}.

%%http://rab.ict.pwr.wroc.pl/dydaktyka/instrukcje/ros_instrukcja.pdf
\begin{figure}[htbp]
 \begin{center}
  \includegraphics[width=\textwidth]{wykorzystane_narzedzia/materialy/ros_structure}    
  \caption{Schemat projektu na platformie \textsf{ROS} \cite{pwrinstrukcja}}
  \label{fig:ros_struktura}
 \end{center}
\end{figure} 

\subsubsection{Pakiety (\textit{packages})}
%http://wiki.ros.org/ROS/Concepts
Pakiety s� g��wn� jednostk� s�u��c� do organizacji oprogramowania. Mog� zawiera�: w�z�y, biblioteki zale�ne od \textsf{ROS}a, zbiory danych, pliki konfiguracyjne lub inne elementy u�ytecznie powi�zane ze sob�. Pakiety s� najmniejsz� jednostk� budulcow� systemu \textsf{ROS}, czyli najmniejsz� rzecz�, kt�r� mo�na zbudowa� i~udost�pni�. Ka�dy z~nich zawiera dokumentacj� (\textit{manifest}) opisan� w~pliku \texttt{package.xml}.

\subsubsection{W�z�y (\textit{nodes})}
%http://wiki.ros.org/ROS/Concepts
W�z�y s� wykonywalnymi instancjami program�w �rodowiska \textsf{ROS}.
System sterowania robotem zazwyczaj sk�ada si� z~wielu w�z��w. Przyk�adowo dla jednego robota mobilnego mo�na wyr�ni� kilka w�z��w odpowiadaj�cych za r�ne funkcjonalno�ci: skaner laserowy, silniki, lokalizacj�, planowanie ruchu itd.
Wyr�nia si� r�wnie� w�ze� nadrz�dny \texttt{master}, kt�ry odpowiada za poprawno�� komunikacji mi�dzy wszystkimi w�z�ami systemu.

\subsubsection{Tematy (\textit{topics}) i wiadomo�ci (\textit{messages})}
%http://wiki.ros.org/ROS/Concepts
W�z�y porozumiewaj� si� ze sob� poprzez przesy�anie wiadomo�ci. Wiadomo�� jest prost� struktur� danych, w~kt�rej znajdowa� si� mog� pola r�nych typ�w prostych oraz tablic typ�w prostych. Co istotne, struktura mo�e mie� zagnie�d�on� budow�, czyli sk�ada� si� z~dowolnej liczby zagnie�d�onych struktur i~tablic. Wiadomo�ci s� przesy�ane poprzez system transportowy zorganizowany na zasadzie nadawca/odbiorca (\textit{publisher/subscriber}).

W�ze� wysy�a wiadomo�� poprzez opublikowanie jej w danym temacie. Temat jest nazw� s�u��c� do identyfikacji zawarto�ci wiadomo�ci. W�ze� odbieraj�cy �ledzi wybrane tematy i~otrzymuje wiadomo�ci wysy�ane do nich. Mo�e by� wiele r�wnoczesnych nadawc�w i~odbiorc�w dla jednego tematu oraz jeden w�ze� mo�e publikowa� i~subskrybowa� wiele temat�w. Poszczeg�lne w�z�y uczestnicz�ce w~komunikacji nie s� �wiadome istnienia innych. Odpowiada to idei roz��czenia produkcji danych od ich konsumpcji.

\subsubsection{Us�ugi (\textit{services})}
%http://wiki.ros.org/ROS/Concepts
Us�ugi udost�pniaj� inny mechanizm komunikacji pomi�dzy w�z�ami. Us�uga jest zdefiniowana poprzez par� struktur wiadomo�ci: jedn� dla ��dania, drug� dla odpowiedzi. W~przeciwie�stwie do prostego przesy�ania wiadomo�ci, kt�re przekazywane s� w~jednym kierunku na zasadzie wiele-do-wielu, us�ugi pozwalaj� na interakcj� ��danie/odpowied�. Takie podej�cie jest cz�sto wymagane w rozproszonym systemie. W�ze�-serwer oferuje us�ug� o danej nazwie, natomiast w�ze�-klient korzysta z~niej poprzez zg�oszenie wiadomo�ci ��dania, a~nast�pnie oczekiwanie na odpowied�. 

\subsubsection{Worki (\textit{bags})}
%http://wiki.ros.org/ROS/Concepts
Worki s� formatem zapisu i~odtwarzania informacji zawartych w wiadomo�ciach. Jest to istotny mechanizm s�u��cy do przechowywania danych, kt�re mog� by� trudne do zebrania np. danych sensorycznych. Umo�liwiaj� r�wnie� testowanie program�w dla jednolitego, uprzednio zdefiniowanego, zestawu danych.
\section{Qt}
\begin{figure}[htbp]
 \centering
 \includegraphics[width=0.4\textwidth]{wykorzystane_narzedzia/materialy/QT.png}
 \label{fig:QTlogo}
\end{figure}
Qt jest to wieloplatformowy zestaw narz�dzi i bibliotek dedykowany dla j�zyka C++. �rodowisko to jest dost�pne dla takich platform jak X11 (m. in. GNU/LINUX, BSD, Solaris), Windows, Mac OS X oraz dla urz�dze� wbudowanych opartych na linuksie, Windows CE, Symbian czy Android. 
Podstaw� dzia�ania bibliotek Qt jest mechanizm slot�w i sygna��w, kt�re zapewniaj� obs�ug� wszelkich zdarze� w obr�bie aplikacji i s�u�y komunikacji pomi�dzy obiektami. Odbywa si� to w spos�b nast�puj�cy, podczas dzia�ania naszej aplikacji pewne specyficzne akcje, predefiniowane b�d� dodane przez u�ytkownika, emituj� sygna� zawieraj�cy informacje o danym zdarzeniu. Slot natomiast jest to funkcja, kt�ra zostaje wywo�ana jako reakcja na wcze�niej przypisany jej konkretny sygna�. Znacz�c� cz�� pracy wykonuje MOC (Qt Meta Object Compiler), generuj�cy kod odpowiedzialny za obs�ug� wszystkich mechanizm�w bibliotek Qt, z kt�rych w danej aplikacji korzystamy. Dzi�ki temu tw�rcy tego rozwi�zania w bardzo du�ym stopniu upro�cili oraz przyspieszyli procedur� pisania kodu przez programist�.
\newline
Wa�nym przy wyborze narz�dzi do zaimplementowania prostego GUI na potrzeby realizowanego projektu by� fakt, i� ca�y interfejs symulatora Gazebo zosta� oparty w�a�nie o omawiane tutaj biblioteki graficzne. Dodatkowo posiada on do�� du�e wsparcie dla tworzenia w�asnych plugin�w zintegrowanych z owym �rodowiskiem, co znacz�co wp�yn�o na komfort korzystania z naszej aplikacji. Nale�y r�wnie� wspomnie� o prostocie wykorzystywania tego rozwi�zania oraz dobrej dokumentacji zar�wno je�eli chodzi o same biblioteki Qt oraz ich sk�adowe jak i poradniki dotycz�ce tworzenia plugin�w do symulatora Gazebo i komunikacji mi�dzy nimi. 
Pos�uguj�c si� owymi bibliotekami w wersji Qt4 i ich podstawowymi klasami stworzono kilka prostych okien, w kt�rych mie�ci si� niemal�e ca�o�� funkcjonalno�ci naszego projektu. W ten spos�b stworzona zosta�a r�wnie� belka z przyciskami zakotwiczona w interfejsie symulatora a tak�e wspomniane wcze�niej okna daj�ce u�ytkownikowi mo�liwo�� kontrolowania symulacji czy zarz�dzania naszymi modelami robot�w b�d� sali. Owe okna a tak�e dostarczona przez nie funkcjonalno�� jest szczeg�owo opisana w dalszej cz�ci raportu.

\section{Inventor}

\begin{figure}[htbp]
\centering
\includegraphics[width=0.4\textwidth]{wykorzystane_narzedzia/materialy/inventor.jpg}
\end{figure}
�rodowisko Autodesk Inventor Professional jest �rodowiskiem in�ynierskim pozwalaj�cym na wytwarzanie wirtualnych modeli 3D rzeczywistych obiekt�w. Inventor w swojej gamie posiada narz�dzia pozwalaj�ce na szybkie wytwarzanie bry� o precyzyjnie okre�lonym kszta�cie i wymiarach. W�a�nie dzi�ki takim mo�liwo�ciom zosta� wykorzystany przy tworzeniu modeli przedmiot�w umieszczonych w sali. 
Budowa modelu konkretnego przedmiotu rozpoczyna si� tu od si� od narysowania kszta�tu przedmiotu w postaci szkicu 2D, a nast�pnie wyci�gni�cia tego szkicu do bry�y tr�jwymiarowej. Aby rozwija�, wytworzon� w ten spos�b, bry�� tworzy si� kolejne szkice p�askie i r�wnie� si� je wyci�ga. Dzi�ki takiemu rozwi�zaniu mo�emy rozwija� nasz model o kolejne elementy jego wygl�du. Po zako�czeniu podstawowego wyci�gania bry� mamy do dyspozycji takie narz�dzia jak np. fazowanie kraw�dzi, wiercenie otwor�w, zmiana materia��w wykorzystanych do produkcji przedmiotu i wiele innych. Dzi�ki tym funkcjonalno�ciom mo�na doprowadzi� model do stanu, w kt�rym b�dzie on idealnie odwzorowywa� prawdziwy przedmiot zar�wno kszta�tem jak i wygl�dem. Oczywi�cie Inventor Professional, jako profesjonalne narz�dzie in�ynierskie, posiada szereg zaawansowanych funkcji do badania modelu. S� to mi�dzy innymi obliczenia bezw�adno�ci bry�, wytrzyma�o�ci ram no�nych, wykrywanie kolizji w zaawansowanych uk�adach kinematycznych, tworzenie rysunk�w pogl�dowych, monta�owych czy przekroj�w itd. Z racji prostoty modeli przedmiot�w w sali znaczna wi�kszo�� z tych narz�dzi nie zosta�a u�yta przy tworzeniu modeli do projektu. 

\section{Blender}

\begin{figure}[htbp]
 \centering
 \includegraphics[width=0.4\textwidth]{wykorzystane_narzedzia/materialy/blender.jpg}
 \label{fig:blender}
\end{figure}

Blender jest bezp�atnym narz�dziem do tworzenia grafiki 3D. Umo�liwia projektowanie oraz renderowanie zar�wno statycznych modeli, jak i animacji, czy gier.

Oprogramowanie to posiada pot�ne mo�liwo�ci, dzi�ki czemu z powodzeniem mo�e konkurowa� z profesjonalnymi, p�atnymi programami takimi jak 3DS Max, czy Maya. Dzi�ki du�ej popularno�ci i ogromnej rzeszy u�ytkownik�w posiada wsparcie spo�eczno�ci. W po��czeniu z licencj� open-source i mo�liwo�ci� edycji kodu daje on niemal nieograniczone mo�liwo�ci przy tworzeniu wszelkiego rodzaju plugin�w. Dzi�ki mo�liwo�ci u�ywania skrypt�w w Pythonie jego funkcjonalno�� mo�na rozszerza� i automatyzowa�.

W przeciwie�stwie do Inventora, modele od pocz�tku i w ca�o�ci s� projektowane w widoku 3D. Wynika to z faktu, i� Blender znajduje zastosowania g��wnie przy tworzeniu grafiki wirtualnej, a nie przy projektach technicznych. W takim przypadku tego typu podej�cie do projektowania lepiej si� sprawdza.
Blender umo�liwia eksport do formatu Collada (.dae). Funkcja ta by�a kluczowa z powodu konieczno�ci wykorzystania tego w�a�nie formatu przy tworzeniu modeli do symulacji w Gazebo.

\section{Zarz�dzanie projektem}

Zarz�dzanie rozwojem projektu by�o realizowane zgodnie z metodologi� Scrum. Zak�ada ona prac� w kr�tkich okresach zwanych Sprintami, w kt�rych to realizowane s� kolejne zadania wybrane przez zesp�, a po jego zako�czeniu zesp� powinien m�c dostarczy� kolejne dzia�aj�ce funkcjonalno�ci wchodz�ce w sk�ad finalnego produktu. W przypadku naszego projektu Sprinty trwa�y dwa lub cztery tygodnie, a w czasie realizacji projektu odby�o si� pi�� takich iteracji.

Dzi�ki takiemu podej�ciu do organizacji pracy, uda�o si� wyeliminowa� b��dne za�o�enia i wyklarowa� optymaln� architektur� tworzonego systemu. Mi�dzy innymi zrezygnowano z konfiguracji �rodowiska Gazebo poprzez zewn�trzne skrypty, na rzecz wykorzystania plugin�w udost�pnionych przez to �rodowisko, o kt�rych zesp� dowiedzia� si� w trakcie trwania drugiego Sprintu, co spowodowa�o radykaln� zmian� za�o�e� podczas planowania kolejnego Sprintu. 

\subsection{Git}
Git jest systemem kontroli wersji, kt�ry bardzo u�atwia zespo�ow� prac� nad projektem, kt�rego g��wn� warto�ci� wytworzon� jest kod �r�d�owy. 

Repozytorium projektu wirtualnego laboratorium L1.5 zosta�o umieszczone w serwisie Github. Wyb�r ten by� podyktowany darmowym dost�pem narz�dzia oraz mnogo�ci� dodatkowych mo�liwo�ci udost�pnionych przez serwis. 
W celu �atwego zarz�dzania wszelkimi tre�ciami wytworzonymi w projekcie, za�o�ono grup� projektow� (\url{https://github.com/ProjektPrzejsciowy}), w kt�rej znajduj� si� wszelkie repozytoria, takie jak: 
\begin{itemize}
	\item projekt\_przejsciowy -- repozytorium z g��wn� cz�ci� projektu, dodatkami do �rodowiska Gazebo,
	\item models -- repozytorium zawieraj�ce modele pozwalaj�ce odwzorowa� laboratorium L1.5.
\end{itemize}

\subsection{Trello}

Trello to bardzo proste narz�dzie wspomagaj�ce planowanie i zarz�dzanie projektami. Mo�liwe jest stworzenie w nim tablicy, do kt�rej mo�emy dodawa� listy z kartami, kt�re z kolei mo�na dowolnie edytowa� i przenosi� mi�dzy listami. 
Dzi�ki takiej funkcjonalno�ci mo�liwe by�o wykorzystanie Trello jako tablicy Scrum'owej w wersji online. Dodatkowo przy pomocy wtyczki ,,Scrum for Trello'' by�o mo�liwe �atwe przegl�danie estymowanego oraz ,,zu�ytego'' czasu dla poszczeg�lnych zada�.
Wykorzystanie Trello bardzo pomog�o~w realizacji projektu, dzi�ki temu narz�dziu cz�onkowie zespo�u mieli mo�liwo�� wzajemnego egzekwowania wykonywanych przez siebie zada�.  

Stworzona podczas realizacji projektu tablica znajduje si� pod adresem \url{https://trello.com/b/WGkwX6TM/backlog}. Wed�ug danych w niej zebranych, wszystkie zadania zwi�zane z powstawaniem projektu zosta�y wyestymowane na 926 roboczogodzin, a~w~rzeczywisto�ci uda�o si� zrealizowa� je w 592 roboczogodziny.



\chapter{Opis systemu}

Symulator robot�w jest doskona�ym narz�dziem dla ka�dej osoby zajmuj�cej si� robotyk�. Pozwala szybko przetestowa� r�ne algorytmy i konstrukcje oraz skomplikowane systemy realizuj�ce niecodzienne scenariusze. Jednym z takich narz�dzi jest darmowy program Gazebo, przeznaczony do tworzenia dok�adnych i efektywnych symulacji robot�w dzia�aj�cych w z�o�onych �rodowiskach. Posiada zaawansowany silnik fizyki, wysokiej jako�ci grafik� oraz wygodne i programowalne interfejsy.
\\ \\
To powy�ej to pr�ba t�umaczenia opisu poni�ej ze strony Gazebo :D
\\ \\
Robot simulation is an essential tool in every roboticist's toolbox. A well-designed simulator makes it possible to rapidly test algorithms, design robots, perform regression testing, and train AI system using realistic scenarios. Gazebo offers the ability to accurately and efficiently simulate populations of robots in complex indoor and outdoor environments. At your fingertips is a robust physics engine, high-quality graphics, and convenient programmatic and graphical interfaces. Best of all, Gazebo is free with a vibrant community.

\section{Okienka}
Funkcjonalno�ci + Implementacja
\subsection{Konfiguracja �wiata}
\begin{figure}[htbp]
 \centering
 \includegraphics[width=1.0\textwidth]{opis_systemu/materialy/konfiguracja_swiata.jpeg}
 \label{fig:konfiguracja_swiata}
 \caption{Okno konfiguracji �wiata}
\end{figure}

Okno konfiguracji �wiata pozwala zmodyfikowa� zawarto�� sceny. Domy�lnie �adowany jest jedynie ground plane, czyli bazowe pod�o�e, na kt�rym umieszczane s� modele. Okno umo�liwia szybkie rozpocz�cie pracy z wybranymi modelami. U�ytkownik ma mo�liwo�� wyboru sali z listy dost�pnej w lewej cz�ci okna. Lista tworzona jest na podstawie plik�w konfiguracji sali z rozszerzeniem .txt znajduj�cych si� w folderze worlds, w katalogu �r�d�owym Gazebo. Przyk�adowy zestaw danych ma nast�puj�c� posta�:
\begin{lstlisting}[language=bash]
smietnik 4.6 1.3 0 0 0 0
szafa 1.5 -3.6 0 0 0 0
kaloryfer -4.5 1.6 0 0 0 1.570796
kaloryfer -4.5 -2.8 0 0 0 1.570796
stol -2.4 2.5 0 0 0 0
stol -4 2.5 0 0 0 0
stol -0.7 -3.4 0 0 0 0
krzeslo -1.7 -2.6 0 0 0 0
krzeslo -1 -2.6 0 0 0 0
krzeslo -0.1 -2.6 0 0 0 0
\end{lstlisting}
W ka�dej linii dodawany jest jeden z dost�pnych modeli. Trzy pierwsze liczby oznaczaj� pozycj� modelu na scenie (X, Y, Z). Pozosta�e okre�laj� orientacj� wok� ka�dej z osi.

Naci�niecie przycisku "Wczytaj" powoduje wyczyszczenie sceny i za�adowanie wybranej sali. Przycisk "Wyczy��" usuwa wszystkie statyczne modele.

Prawa strona okna konfiguracji �wiata pozwala wybra� liczb� robot�w Pioneer, z kt�rymi u�ytkownik chce rozpocz�� prac�. Liczba pozycji w lewym polu okre�la ile robot�w jest dost�pnych, natomiast liczba pozycji w prawym polu definiuje liczb� robot�w na scenie. Zawarto�� p�l mo�na modyfikowa� za pomoc� przycisk�w "Dodaj" i "Schowaj".
% Krzysztof Kwieci�ski
\subsection{Zarz�dzanie robotami}
Okienko \texttt{Zarz�dzanie robotami} umo�liwia ustawianie pozycji oraz orientacji wybranego robota na scenie.

Zak�adki umo�liwiaj� wyb�r Pioneera, kt�rego pozycj� chcemy zmieni�. Aktywne s� tylko zak�adki skojarzone z~robotami dodanymi aktualnie do �wiata. 
Po wyborze zak�adki mo�liwe jest podanie nowych wsp�rz�dnych $XY$ oraz orientacji $\theta$ robota. W~polach odpowiedzialnych za pozycj� i~orientacj�, kolor informuje o~poprawno�ci wprowadzonych danych -- zielony oznacza poprawnie wype�nione pola, natomiast czerwony b��dnie. 
Przyci�ni�cie przycisku \texttt{Ustaw} powoduje ustawienie Pioneera zgodnie z~��daniem. Przycisk \texttt{Reset} przenosi robota do punktu $(0, 0)$ i~ustawia jego orientacj� na~$0$. 

Po zmianie zak�adki lub ustawieniu/zresetowaniu pozycji, w~polach wy�wietlane s� aktualne wsp�rz�dne robota.

\begin{figure}[htbp]
 \centering
 \includegraphics[]{opis_systemu/materialy/zarzadzanie_robotami.jpeg}
 \label{fig:zarzadzanie_robotami}
 \caption{Okno zarz�dzania robotami}
\end{figure}

\subsubsection{Implementacja}
Klasa odpowiadaj�ca za ca�e okienko to \texttt{RobotManagementWindow}. Dziedziczy ona po klasie \texttt{QDialog}. 

Sloty odpowiadaj�ce za to, co dzieje si� w~oknie w~momencie dodania lub schowania robota, to:
\begin{lstlisting}[language=C++, numbers=none]
public slots:
	void onAddRobot(int id);
	void onHideRobot(int id);
\end{lstlisting}
W~momencie otrzymania odpowiednich sygna��w aktywowana b�d� dezaktywowana jest stosowna zak�adka, a~tak�e ustawiane s� warto�ci w~polach.

Wy�wietlanie pozycji i~orientacji robota jest mo�liwe dzi�ki subskrybowaniu odpowiedniego tematu i~wywo�aniu funkcji aktualizuj�cej pozycj� robota:
\begin{lstlisting}[language=C++, numbers=none]
this->sub = node->Subscribe("~/pose/info", 
                            &RobotManagementWindow::OnPoseMsg, this);
\end{lstlisting}

Klasa odpowiadaj�ca za wygl�d pojedynczej zak�adki oraz g��wn� funkcjonalno�� okienka to \texttt{RobotManagementTab}. Dziedziczy ona po klasie \texttt{QWidget}. Jej kluczowe metody to:
\begin{lstlisting}[language=C++, numbers=none]
private slots:
	void on_pushButtonUstaw_clicked();
	void on_pushButtonReset_clicked();
\end{lstlisting}
Odpowiadaj� one za ustawienie b�d� zresetowanie pozycji robota. Przyci�ni�cie kt�rego� z~przycisk�w skutkuje wywo�aniem us�ugi \texttt{/gazebo/set\_model\_state}.














\subsection{Wyniki symulacji}

\input{opis_systemu/wyniki_symulacji/funkcjonalnosc}
\input{opis_systemu/wyniki_symulacji/implementacja}
\section{Pluginy}
\subsection{GUI}
\subsection{World}
\section{Integracja Gazebo z ROS}
G��boka integracja Gazebo z ROS jest najwi�ksz� zalet� tego symulatora. Umo�liwia j� zestaw dostarczanych plugin�w do ROS w pakiecie gazebo\_ros\_pkgs.
Podstawowe funkcjonalno�ci dostarczane razem z pakietem:
\begin{itemize}
\item pomimo integracji Gazebo pozostaje nadal samodzielnym systemem,
\item mo�liwo�� zbudowania pakiet�w Gazebo w catkin,
\item zmniejszenie ilo�ci kodu potrzebnego do symulacji,
\item udost�pnia u�ytkownikowi szereg us�ug i temat�w ROS do zarz�dzania symulacj� (wymienione na Rysunku \ref{fig:ros_integracja})
\end{itemize}

\begin{figure}[htbp]
 \begin{center}
  \includegraphics[width=\textwidth]{opis_systemu/materialy/gazeborosapi}   
  \caption{Schemat prezentuj�cy funkcje udost�pniane przez gazebo\_ros\_pkgs}
  \label{fig:ros_integracja}
 \end{center}
\end{figure} 

\subsection{Kompilacja pakiet�w Gazebo przy u�yciu catkin}
Jak zaznaczono wcze�niej, mo�liwe jest bezpo�rednie u�ycie systemu catkin do budowania pakiet�w napisanych dla Gazebo. Wymaga�o to stworzenia catkin workspace -- katalogu, w kt�rym budowane s� pakiety catkin. Stworzone przez nas pluginy Gazebo zosta�y zebrane i tak skonfigurowane, �e tworz� pakiet kompatybilny z ROS.

Po uruchomieniu, Gazebo tworzy w�asny w�ze� do komunikacji z ROS. Dzi�ki kompilacji poprzez catkin, pluginy Gazebo maj� dost�p do ROSa, bezpo�rednio w kodzie mo�emy odwo�ywa� do jego funkcji. Aby si� o tym upewni�, w World plugin umieszczono poni�szy kod, kt�ry zatrzymuje dzia�anie Gazebo w przypadku braku komunikacji z ROS.

\begin{lstlisting}[language=c]
...                                                                                   
if ( !ros::isInitialized() )
{
	ROS_FATAL_STREAM("A ROS node for Gazebo has not been initialized");
	return;
}
...
\end{lstlisting}

\subsection{Uruchamianie Gazebo przy u�yciu narz�dzi z ROS}
Dzi�ki u�yciu catkin'a mo�emy uruchamia� Gazebo za pomoc� narz�dzia roslaunch, kt�re pozwala na automatyczne wywo�ywanie w�z��w ROS oraz wst�pn� konfiguracj�. U�atwia to start skonfigurowanego do pracy �rodowiska, co sprowadza si� do jednego polecenia:
\begin{verbatim}
$ roslaunch projekt_przejsciowy l15.launch
\end{verbatim}
Aby to umo�liwi�, przygotowano odpowiedni plik l15.launch w podkatalogu stworzonego pakietu Gazebo $launch$. Plik launch mo�e przy uruchomieniu wczytywa� plik world z~konfiguracj� Gazebo lub umieszcza� modele w odpowiednich miejscach.

\subsection{Udost�pnione funkcjonalno�ci do komunikacji ROS}
Jak wida� na Rysunku \ref{fig:ros_integracja}, Gazebo udost�pnia szereg funkcjonalno�ci kt�rych mo�na u�y� do integracji z ROS. 

Dla projektu najwi�ksze znaczenie mia�y pluginy wysy�aj�ce dane z czujnik�w na robocie, takich jak skaner laserowy czy kamera oraz umo�liwiaj�ce sterowanie nap�dem robota. Zosta�y one zintegrowane z modelem Pioneera.

W�ze� tworzony przez Gazebo publikuje informacje na zewn�trz za pomoc� temat�w, np. \~/model\_states, wysy�aj�cy informacje o stanie modeli aktualnie u�ywanych w symulacji. W�ze� dostarcza r�wnie� us�ugi ROS'a, takie jak \~/spawn\_urdf\_model do �adowania modeli robot�w w formacie URDF, czy \~/delete\_model do usuwania modeli. Mog� zosta� one wykorzystane przez u�ytkownika naszego systemu do kontroli symulacji i zbierania danych symulacyjnych.
\section{Modele}

Modele u�yte do stworzenia laboratorium L1.5 zosta�y wykonane w programach Inventor Professional oraz Blender.

Elementy wyposa�enia wn�trza takie jak sto�y, krzes�a, grzejniki, szafa oraz �mietnik zaprojektowane zosta�y w Inventorze, natomiast sama sala (�ciany) w programie Blender. Wszystkie stworzone elementy opracowane zosta�y w oparciu o pomiary rzeczywistych mebli, przedmiot�w i �cian w sali L1.5 w budynku C-16. Nast�pnie wszystkie elementy zosta�y przekonwertowane przy u�yciu programu Blender do formatu Collada (.dae), aby mo�liwe by�o wykorzystanie ich w Gazebo.

Modele zosta�y r�wnie� posk�adane w Blenderze w celu wyrenderowania widok�w z r�nych miejsc sali:

\begin{center}

	\includegraphics[width=12cm]{opis_systemu/materialy/R4.jpg}
	
	\includegraphics[width=12cm]{opis_systemu/materialy/R1.jpg}

\end{center}

Aby stworzone modele mog�y by� u�yte w Gazebo, konieczne by�o jeszcze napisanie plik�w konfiguracyjnych "model.config" oraz "model.sdf".
\\
Plik "model.config" zawiera takie informacje jak:
\begin{itemize}
\item nazwa obiektu
\item lista wersji pliku .sdf
\item informacje o autorze
\end{itemize}
W pliku "model.sdf" przechowywane s� informacje na temat:
\begin{itemize}
\item w�a�ciwo�ci fizycznych obiektu (masa, momenty bezw�adno�ci)
\item statyczno�ci obiektu (mo�liwo�ci poruszenia przez inny obiekt)
\item geometrii obiektu wyko�ystywanej przy kolizji z innymi obiektami
\item wizualizacji obiektu w symulacji
\end{itemize}
W przypadku modeli z�o�onych, takich jak np. roboty, plik "model.sdf" mo�e zawiera� r�wnie�:
\begin{itemize}
\item informacje o w�a�ciwo�ciach fizycznych poszczeg�lnych cz�on�w obiektu
\item parametry po��cze� kinematycznych mi�dzy cz�onami
\item pluginy wyko�ystywane w modelu
\item dane o czujnikach i ich umieszczeniu
\end{itemize}
Do stworzenia modelu robota Pioneer wykorzystano pliki zawarte w Gazebo, jednak aby mo�na by�o go u�y� w symulacjach konieczne by�o zaimplementowanie jego funkcjonalno�ci w pliku .sdf. W tym celu dodano pluginy do obs�ugi nap�d�w oraz sensor�w, kt�re zosta�y dodane do modelu.

Dzi�ki tym zabiegom podczas dodawania robota w symulacji tworzone s� topici pozwalaj�ce na komunikacj� z modelem robota.
\chapter{Testy}

W ramach test�w systemu przygotowano �rodowisko zgodnie z instrukcj� u�ytkownika (zawarta w dodatku A). Po uruchomieniu systemu symulacyjnego wykonano czynno�ci sprawdzaj�ce dzia�anie podstawowych funkcji oprogramowania, wymaganych do bezproblemowego u�ytkowania go.

W pierwszej kolejno�ci sprawdzono poprawno�� wczytywania si� modelu laboratorium, oraz umieszczania i chowania robot�w ze sceny, test ten wypad� pomy�lnie.
Kolejnym wa�nym aspektem jest to czy w systemie ROS, po dodaniu robota pojawiaj� si� odpowiednie tematy. Dla robota 'pioneer\_1' s� one nast�puj�ce:

\begin{itemize}
\item \begin{verbatim}/pioneer_1/RosAria/cmd_vel\end{verbatim}
\item \begin{verbatim}/pioneer_1/RosAria/pose \end{verbatim}
\item \begin{verbatim}/pioneer_1/camera/camera_info\end{verbatim}
\item \begin{verbatim}/pioneer_1/camera/image_raw\end{verbatim}
\item \begin{verbatim}/pioneer_1/camera/parameter_descriptions\end{verbatim}
\item \begin{verbatim}/pioneer_1/camera/parameter_updates\end{verbatim}
\item \begin{verbatim}/pioneer_1/scan\end{verbatim}
\end{itemize}


\begin{figure}[htbp]
 \centering
 \includegraphics[width=0.8\textwidth]{testy/test1.png}
 \label{fig:docker}
\end{figure}


Nast�pnie sprawdzono mo�liwo�� zarz�dzania pozycj� robot�w z poziomu okienka 'Zarz�dzanie robotami'. Przeprowadzone testy wykaza�y, �e pozycja robota ustawiana jest w spos�b odpowiedni, a b��dnie wprowadzone dane s� odpowiednio sygnalizowane. Jedynym problemem okaza� si� brak weryfikacji czy miejsce w kt�rym chcemy postawi� robota jest wolne i czy nie wyl�duje on np. na innym robocie.


Ostatni� grup� funkcjonalno�ci jest mo�liwo�� rejestracji pozycji robota i zapisanie jej do pliku. W celu jej weryfikacji do systemu ROS opublikowano dwie wiadomo�ci,  kt�re w efekcie powinny umo�liwi� narysowanie robotem �semki:
\begin{itemize}
\item \begin{verbatim}rostopic pub /pioneer_1/RosAria/cmd_vel \
geometry_msgs/Twist -- '[0.5, 0.0, 0.0]' '[0.0, 0.0, 0.2]' \end{verbatim}
\item \begin{verbatim}rostopic pub /pioneer_1/RosAria/cmd_vel \
geometry_msgs/Twist -- '[0.5, 0.0, 0.0]' '[0.0, 0.0, -0.3]' \end{verbatim}
\end{itemize}

Po wykonaniu powy�szych polece� robot pokona� zamierzon� tras�, a otrzymane wykresy ruchu zgadza�y si� z rzeczywisto�ci�.

\begin{figure}[htbp]
 \centering
 \includegraphics[width=0.6\textwidth]{testy/test3.png}
 \label{fig:docker}
\end{figure}


\section{Przygotowanie do laboratorium}

W celu sprawdzenia mo�liwo�ci korzystania z modu�u RosAriaDriver, napisano prosty program umo�liwiaj�cy poruszanie si� po kwadracie, z wykorzystaniem napisanej w Pythonie biblioteki drive\_simulation.py dost�pnej w przygotowanym kontenerze dockera.
\begin{verbatim}
from drive_simulation import RosAriaDriver
p = RosAriaDriver('pioneer_1')
for i in range(4):
    p.SetSpeedLR(0.2, 0.2, 5)
    p.SetSpeed(0, 0.2, 5)
\end{verbatim}

W ten spos�b uda�o si� uzyska� przejazd po �cie�ce przypominaj�cej kwadrat.

\begin{figure}[htbp]
 \centering
 \includegraphics[width=0.6\textwidth]{testy/test4.png}
 \label{fig:docker}
\end{figure}

\chapter{Wnioski}
\section{Serwer}
Pocz�tkowo, r�wnocze�nie z wersj� aplikacji wykorzystuj�c� Dockera, rozwijana by�a wersja aplikacji dzia�aj�ca na serwerze. Wed�ug za�o�e� Gazebo, ROS oraz Ubuntu by�y zainstalowane na jednym komputerze a klienci ��czyli si� z nim za pomoc� klienta X11. Podej�cie to zapewnia�o zar�wno kompatybilno�� z systemem klienta dzia�aj�cym pod Unixem przy wykorzystaniu systemowego terminalu jak i systemem Windows przy wykorzystaniu np. Putty i Xminga. Jednak�e napotkane problemy sprawi�y, i� zaprzestano dalszego rozwoju aplikacji w wersji serwerowej.

Problemy kt�re wykluczy�y tak� wersj� aplikacji to:
\begin{itemize}
	\item Gazebo �le wsp�pracuje ze �rodowiskiem graficznym X11. Zar�wno na kartach graficznych AMD jak i NVIDIA wyst�powa� problem z przes�aniem obrazu. �eby aplikacja dzia�a�a u~klienta poprawnie trzeba by�o wy��czy� sprz�tow� akceleracj� GPU co przy �rodowisku 3D Gazebo znacznie wp�ywa�o na wydajno��.
	\item Do komfortowej pracy z systemem potrzeba co najmniej 15 klatek na sekund�. Taka ilo�� obrazu do przes�ania generowa�a bardzo du�� ilo�� danych, kt�ra musia�a przej�� przez sie�. Dla zapewnienia 15 klatek na sekund� dla pojedynczego klienta serwer wysy�a� ok 300Mb danych, tak� ilo�� musia� te� odebra� komputer klienta. Korzystaj�c z sieci WiFi Politechnki Wroc�awskiej nie uda�o si� osi�gn�� wi�cej ni� $0.5$ klatki na sekund�. 
\end{itemize}
\appendix
\documentclass[10pt, a4paper]{article}

%Preambuła dokumentu
\usepackage{graphicx}       % pakiet graficzny, umożliwiający m.in.
                            % import grafik w formacie eps
\usepackage{rotating}       % pakiet umożliwiający obracanie rysunków
\usepackage{subfigure}      % pakiet umożliwiający tworzenie podrysunków
\usepackage{epic}           % pakiet umożliwiający rysowanie w środowisku latex
\usepackage{listings}       % pakiet dedykowany zrodlom programow
\usepackage{verbatim}       % pakiet dedykowany rozmaitym wydrukom tekstowym
\usepackage{amssymb}        % pakiet z rozmaitymi symbolami matematycznymi
\usepackage{amsmath}        % pakiet z rozmaitymi środowiskami matematycznymi
\usepackage[polish]{babel}  % pakiet lokalizujący dokument w języku polskim
\usepackage[OT4]{fontenc}
\usepackage[utf8]{inputenc} % w miejsce utf8 można wpisać latin2 bądź cp1250,
                            % w zależności od tego w jaki sposób kodowane są 
                            % polskie znaki diakrytyczne przy wprowadzaniu 
                            % z klawiatury.

\usepackage[draft]{prelim2e}% informacja w stopcje o statusie dokumentu 
                            % (draft-szkic lub final-wersja ostateczna) 

\textwidth      16cm
\textheight     25.5cm
\evensidemargin -3mm
\oddsidemargin  -3mm
\topmargin      -20mm


% deklaracje wymagane przez funkcję drukującą tytuł dokumentu:
%
\author{Automatyka i Robotyka, specjalność Robotyka\footnote{Wydział Elektroniki, Politechnika Wrocławska, ul. Z. Janiszewskiego 11/17, 50-372 Wrocław} }

 
\title{ Wirtualne laboratorium robotyki - Instrukcja } 

\date{\today}

% Koniec preambuły dokumentu

% Tekst dokumentu

\begin{document}
\maketitle % drukuje tytul, autora i datę zdefiniowaną w preambule
%
%\the\setitem
\def\tablename{Tabela}
%

\section{Wstęp}
\label{sec:wstep} % definicja etykiety rozdziału
W niniejszej instrukcji znajduje się opis czynności potrzebnych do uruchomienia i użytkowania systemu wirtualnego laboratorium.

\section{Przygotowanie systemu do uruchomienia}
\label{sec:przygotowanie}
Do uruchomiania systemu wymagane jest zainstalowanie lokalnie Dockera. W przypadku braku tego pakietu na komputerze należy postąpić zgodnie z instrukcją zamieszoną na stronie Dockera:

\begin{verbatim}
https://docs.docker.com/engine/installation/linux/ubuntu/
\end{verbatim}


Aby nie musieć za każdym razem używać $sudo$ w poleceniach dotyczących Dockera należy po zakończeniu instalacji wykonać polecenie:
\begin{verbatim}
 sudo usermod -aG docker [nazwa twojego użytkownika] 
\end{verbatim}

i potwierdzić hasłem użytkownika. 

Następnie należy pobrać na dysk lokalny obraz systemu znajdujący się w repozytorium DockerHub. Należy pobrać najnowszy obraz wersji beta o nazwie :
\begin{verbatim}
 projektprzejsciowy2016$/docker_lab_beta:latest
\end{verbatim}

Możliwe jest to za pomocą polecenia:
\begin{verbatim}
docker pull projektprzejsciowy2016/docker_lab_beta:latest
\end{verbatim}


\begin{figure}[hbt]
  \setlength{\unitlength}{1.0cm}
  \centering 
  
    \includegraphics[width=12 cm]{./grafika/pull.png}
 
\end{figure}


Przed uruchomieniem obrazu należy zezwolić wszystkim aplikacjom działającym z poziomu Dockera na uruchamianie się w trybie graficznym:
\begin{verbatim}
 xhost +
\end{verbatim}

Powinien pojawić się następujący komunikat:
$"$access control disabled, clients can connect from any host$"$

Uruchomienie obrazu systemu odbywa się poprzez wpisanie w konsoli polecenia (powoduje stworzenie nowego kontenera, na podstawie wywołanego i otwiera go - wszelkie zmiany będą zapisywane w nowym kontenerze) :

\begin{verbatim}
 docker run -it -e DISPLAY=unix$DISPLAY -v=/tmp/.X11-unix:/tmp/.X11-unix:rw \
 projektprzejsciowy2016/docker_lab_beta:latest

\end{verbatim}

W przypadku gdy potrzebujemy współdzielić pewne katalogi lokalne z Dockerem należy  w poleceniu uruchamiającym obraz systemu $"$docker run$"$ skorzystać z opcji -v:
\begin{verbatim}
docker run -it -e DISPLAY=unix$DISPLAY -v=/tmp/.X11-unix:/tmp/.X11-unix:rw \
-v[scieżka_do_katalogu_lokalnego]:[scieżka_gdzie_ma_być_montowany_w_obrazie] \
projektprzejsciowy2016/docker_lab_beta:latest

\end{verbatim}


System ROS wraz z symulatorem - Gazebo, uruchamiany jest poleceniem wykonywanym już z poziomu obrazu Dockera:
\begin{verbatim}
roslaunch projekt_przejsciowy l15.launch &
\end{verbatim} 

Po chwili powinien uruchomić się symulator Gazebo:

\begin{figure}[hbt]
  \setlength{\unitlength}{1.0cm}
  \centering 
  
    \includegraphics[width=12 cm]{./grafika/EkranGlowny.png}

\end{figure}
 Aby przetestować czy wszystko działa poprawnie należy dodać na scenę robota jak w puncie 3.2.


Jeśli potrzebujemy uruchomić drugą konsolę działającą na tym samym kontenerze to należy uruchomić drugą konsolę (można to wykonać skrótem klawiszowym $"$shitf$"$+$"$ctrl$"$+$"$t$"$, a następnie sprawdzamy nazwę kontenera, który mamy już uruchomiony:

 \begin{verbatim}
docker ps -a
\end{verbatim}


Powyższe polecenia wyświetla listę wszystkich dostępnych lokalnie kontenerów, w której można sprawdzić nazwy poszczególnych kontenerów:


\begin{figure}[hbt]
  \setlength{\unitlength}{1.0cm}
  \centering 
  
    \includegraphics[width=12 cm]{./grafika/ps-a.png}

\end{figure}

Na przedstawionej na rysunku liście przykładowe nazwy to goofy\_varahamihira oraz stupefied\_tesla. Podpięcie się konsolą pod otwarty już kontener opiera się o polecenie:


 \begin{verbatim}
docker exec -it Nazwa_Kontenera bash
\end{verbatim}


Uruchamiając kontener po raz kolejny mamy możliwość zachowania historii wykorzystywanych poleceń. Przy uruchamianiu kontenera odwołujemy się do ostatniego otwieranego dockera:
\begin{verbatim}
docker start -i $(docker ps -q -l)
\end{verbatim}



\section{Obsługa systemu}
\subsection{Dodawanie modelu sali}
Na obrazie systemu, w chwili obecnej, znajdują się trzy przykładowe modele świata:
\begin{itemize}
\item model sali L1.5 w budynku C-16
\item dwa dodatkowe modele składające się z ułożonych w pewien sposób mebli 
\end{itemize}

Dodanie jednego z tych światów możliwe jest z poziomu okna "Konfiguracja Świata". Dostep do wspomnianego okna jest z poziomu widoku głównego Gazebo pod przyciskiem umieszczonym z lewej strony listwy na górze ekranu.
\begin{figure}[hbt]
  \setlength{\unitlength}{1.0cm}
  \centering 
  
    \includegraphics[width=12 cm]{./grafika/KOnfiguracjaSwiata.png}

\end{figure}
W oknie tym, po lewej stronie, znajduje się lista dostępnych światów oraz dwa przyciski "Wczytaj" i "Wyczyść". Wybraną salę można dodać wybierając ją z lisy i klikając pierwszy przycisk. Jeśli wczytamy inną salę w momencie, gdy jakaś jest już wczytana, scena zostanie automatycznie wyczyszczona i wczytany aktualnie wybrany świat. Przycisk "Wyczyść$"$ służy do usunięcia wszystkich obiektów ze sceny z wyjątkiem robotów.

 \subsection{Dodawanie modeli robotów}
Z okna konfiguracji świata można również dodawać na scenę roboty (jak narazie dostępne są trzy roboty Pioneer). Służą do tego listy na środku oraz z prawej strony okienka. Lista środkowa zawiera roboty, które są dostępne do dodania, a prawa te, które są już dodane. Aby dodać robota, należy wybrać go z listy środkowej i kliknąć przycisk " Dodaj". Roboty dodane pojawią się na scenie w momencie kliknięcia przycisku $"$Zatwierdź$"$.  W tym samym momencie pojawiają się w systemie ROS topiki związane z dodawanymi robotami. Listę topików można zobaczyć wpisując w terminalu:
\begin{verbatim}
rostopic list
\end{verbatim}
Przycisk $"$Schowaj" umieszcza wybranego robota poza salą w tak zwanym schowku.

Aby przetestować czy Gazebo dobrze współpracuje z systemem ROS należy dodać robota a następnie w konsoli wydać polecenie publikujące w topicu ROSa odpowiadającym za prędkość dodanego robota:
\begin{verbatim}
rostopic  pub /pionner_1/RosAria/cmd_vel geometry_msgs/Twist \
-- '[1.0, 0.0, 0.0]' '[0.0, 0.0, -0.5]
\end{verbatim}

W efekcie robot powinien zacząć poruszać się po okręgu.

\begin{figure}[hbt]
  \setlength{\unitlength}{1.0cm}
  \centering 
  
    \includegraphics[width=12 cm]{./grafika/EkranGlownyZSalaIRobotami.png}

\end{figure}

 \subsection{Resetowanie robotów i ustawianie ich w zadanych miejscach}

Do resetowania i ustawiania w zadanej pozycji robotów służy okno $"$Zarządzanie robotami". Okno podzielone jest na zakładki - każda z nich dotyczy osobnego robota. Umieszczony w każdej zakładce przycisk $"$Reset$"$ ustawia odpowiedniego robota w punkcie (0,0) sali z orientacją równą 0. Pola tekstowe służą do zadawania pozycji (X,Y) i orientacji robota. Wartości z tych pól są wykorzystywane przy używaniu przycisku "Ustaw". W trakcie kliknięcia wartości te są sprawdzane i ich ewentualna niepoprawność jest sygnalizowana czerwonym podświetleniem pola. Dopiero po sprawdzeniu robot przemieszczany jest na ustaloną pozycję.
\begin{figure}[hbt]
  \setlength{\unitlength}{1.0cm}
  \centering 
  
    \includegraphics[width=6 cm]{./grafika/OknoZarzadzaniaRobotami.png}

\end{figure}


\subsection{Wyświetlanie i zapisywanie wyników}
Do wizualizacji i zapisu danych z eksperymentu przygotowane zostało okno $"$Wyniki symulacji$"$. Okno podzielone jest na zakładki, w których można otrzymać podgląd trajektorii wszystkich robotów oraz przebiegi położenia X, Y i orientacji. W pierwszej zakładce są trasy przebyte przez wszystkie roboty a każda kolejna zakładka odnosi się do konkretnego robota prezentując jego położenie i orientację. Przycisk $"$Reset danych$"$ pozwala wyczyścić wykresy i zacząć zbierać je od nowa.  Okno daje możliwość zapisania przebiegów do plików, które znajdują się potem w folderze /root/.  Ważne jest też aby przed rozpoczęciem zbierania danych do konkretnego wykresu za pomocą przycisku z dolnego paska Gazebo $"$PLAY/PAUSE$"$ zatrzymać czas symulacyjny następnie przyciskiem $"$Reset Time$"$ aby wykres był czytelny.
\begin{figure}[hbt]
  \setlength{\unitlength}{1.0cm}
  \centering 
  
    \includegraphics[width=6 cm]{./grafika/Wykres.png}

\end{figure}

\subsection{Wyświetlanie odczytów z czujników}
Aby uzyskać dostęp do pomiarów z czujników należy z paska narzędzi Gazebo wybrać zakładkę $Window$ $>>$ $Topic$ $Visualization$. Uruchomi się okno:
 \begin{figure}[hbt]
  \setlength{\unitlength}{1.0cm}
  \centering 
  
    \includegraphics[width=6 cm]{./grafika/WizualizacjaLista.png}

\end{figure}

Następnie wybieramy topic czujnika, z którego chcemy wyświetlać dane. Dla skanera laserowego wykres wygląda następująco:

 \begin{figure}[hbt]
  \setlength{\unitlength}{1.0cm}
  \centering 
  
    \includegraphics[width=8 cm]{./grafika/Skaner.png}

\end{figure}
\newpage
\subsection{Dodawanie i edycja modeli przeszkód na scenie}
Aby dodać przeszkodę na scenę należy z lewej strony ekranu wejść w zakładkę "Insert", wybrać kliknięciem model i kliknąć na miejsce na scenie, w którym chcemy postawić przeszkodę. Proste modele przeszkód dostępne są z paska narzędzi Gazabo. 
Aby zmienić statyczność przeszkody należy kliknąć na nią prawym przyciskiem myszy (PPM) a następnie z rozwiniętego menu wybrać opcję $"$Edit Model". Gazebo przejdzie do trybu edycji modelu. Wybieramy zakładkę "Model" i zaznaczamy lub odznaczamy opcję $"$Static":

\begin{figure}[hbt]
  \setlength{\unitlength}{1.0cm}
  \centering 
  
    \includegraphics[width=6 cm]{./grafika/Edit.png}

\end{figure}



\end{document}
 

\chapter*{Dodatek A: Instrukcja dla dewelopera}

\bibliographystyle{plabbrv}
\nocite{*}
\bibliography{bibliografia}

\end{document}
