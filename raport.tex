\documentclass[12pt,a4paper,twoside,openright,fleqn]{mwrep}
\usepackage[latin2]{inputenc}
\usepackage[OT4,plmath]{polski}
\usepackage{amsmath}
\usepackage{amssymb}
\usepackage{graphicx}
%\usepackage[a4paper,left=3.5cm,right=2.5cm,top=2.5cm,bottom=2.5cm]{geometry}
\usepackage[width=16cm,height=25cm]{geometry}
\usepackage{fancyvrb}
\usepackage{listings}
%\usepackage{bold-extra}
\usepackage[unicode=true]{hyperref}
\usepackage{color}
\usepackage{cite}
\usepackage{float}

\pagestyle{uheadings}

% Definicje ustalaj�ce wygl�d wydruk�w program�w
\definecolor{darkgray}{rgb}{0.95,0.95,0.95}
\lstset{frame=tb, backgroundcolor=\color{darkgray}}
\lstset{numbers=left, numberstyle=\tiny, stepnumber=2, numbersep=5pt}
\lstset{basicstyle=\ttfamily, keywordstyle=\color{blue}\bfseries}


\hypersetup{
    pdfpagemode=UseOutlines,   % otwiera dokument w trybie jednej strony
    pdfpagelayout=SinglePage,  %
    pdfstartpage=1,            % na podanej stronie
    bookmarksopen=true,        % rozwini�cie zak�adek
    bookmarksopenlevel=1,   % do jakiego poziomu
    colorlinks=true,   % kolorowanie odno�nik�w zamiast ramki wok� nich
    citecolor=cyan,   % kolor odno�nik�w do bibliografii, domy�lnie zielony
    filecolor=red,   % kolor odno�nik�w do lokalnych plik�w, domy�nie magenta
    linkcolor=blue,   % kolor odno�nik�w wewn�trznych, domy�lnie czerwony
    menucolor=green,   % kolor pozycji menu Acrobata, domy�lnie czerwony
    urlcolor=blue,   % kolor odno�nik�w do adres�w internetowych, domy�lnie cyan
                               % DANE DOKUMENTACJI
    pdfauthor={ARR2016},
    pdftitle={Projekt przej�ciowy --- Symulator laboratorium L1.5},
 }

% Definicje ustalaj�ce wygl�d listing�w
\usepackage{xcolor}
\definecolor{bluekeywords}{rgb}{0.13,0.13,1}
\definecolor{greencomments}{rgb}{0,0.5,0}
\definecolor{redstrings}{rgb}{0.9,0,0}
\definecolor{dkgreen}{rgb}{0,0.6,0}
\definecolor{lbcolor}{rgb}{0.9,0.9,0.9}
\lstset{
	language=C++,
    frame=tb,
    tabsize=2, 
	showstringspaces=false, 
    basicstyle=\footnotesize\sffamily\color{black},  
	identifierstyle=\ttfamily\color{black},  
    commentstyle=\color{greencomments}, 
    stringstyle=\color{redstrings}, 
    keywordstyle=\color{bluekeywords}, 
    morekeywords={library, description},
    directivestyle={\color{black}},
    numberstyle=\color[rgb]{0.205, 0.142, 0.73},
    emph={virtual, ::},
    emphstyle={\color{magenta}}
}

                                   % DEFINICJE W�ASNE
\newcommand{\dd}{\text{d}}
\newtheorem{uwaga}{Uwaga}
\newtheorem{twr}{Twierdzenie}


\title{Projekt przej�ciowy --- Symulator laboratorium L1.5}
\author{ARR 2016}
\date{\today}


\begin{document}
\VerbatimFootnotes

\maketitle
\tableofcontents

\chapter{Wst�p}
�rodowisko do symulowania dzia�ania robot�w w warunkach laboratorium L1.5 zosta� wykonany na zaj�ciach projektu przej�ciowego grupy ARR. �rodowisko jest dostarczony w postaci kontenera dockera na kt�rym znajduje si� system operacyjny ubuntu 16.04 ros kinetic oraz gazebo 7.5. 

Celem projektu by�o umo�liwienie przygotowania si� studentom do zaj�� laboratoryjnych poprzez testowanie napisanych przez siebie program�w w �rodowisku zbli�onym do udost�pnianej przestrzeni podczas zaj�� praktycznych. Dostarczony projekt umo�liwia wyb�r laboratorium i robot�w pionier oraz pozwala na dodawanie element�w sceny takich jak przeszkody. Z wykorzystaniem �rodowiska ros mo�na sterowa� robotami. Symulator pozwala r�wnie� na zmian� pozycji robota. % oraz rejestracj� i zapis jego pozycji

Osoba kt�rej zadanie ogranicza� si� b�dzie do wykonania �wicze� laboratoryjnych powinna zapozna� si� z instrukcj� z dodatku A. W celu u�atwienia rozwoju i  modyfikacji projektu stworzony zosta� dodatek B.

\section{Motywacja}

Projekt wirtualnego laboratorium powsta� jako pomoc dydaktyczna do zaj�� laboratoryjnych wykonywanych na robotach Pioneer w �rodowisku ROS realizowanych na Politechnice Wroc�awskiej. Dzi�ki �rodowisku studenci mog� w ramach przygotowania do zaj�� wykona� �wiczenia na robotach Pioneer w systemie ROS dost�pnych w wirtualnym laboratorium. Ponadto, �rodowisko pozwala zainteresowanym na zapoznanie si� ze sposobem budowy �rodowiska jako odizolowanego kontenera Dockera i jego obs�ug�.
\section{Cel projektu}

Celem projektu jest stworzenie wirtualnego laboratorium umo�liwiaj�cego obs�ug� robot�w Pioneer w systemie ROS w �rodowisku odpowiadaj�cym warunkom rzeczywistego laboratorium. Dostarczone �rodowisko umo�liwia wyb�r laboratorium i robot�w Pioneer oraz pozwala na dodawanie element�w sceny, takich jak przeszkody. Sterowanie robotami mo�liwe jest dzi�ki platformie ROS. Symulator pozwala na rejestracj� i zapis pozycji robota w postaci graficznej oraz numerycznej.

�rodowisko dostarczone jest w postaci kontenera Dockera, na kt�rym znajduje si� system operacyjny Ubuntu 16.04, ROS Kinetic oraz Gazebo 7.5. Osoba, kt�rej zadanie ogranicza� si� b�dzie do pracy z systemem (np. w celu wykonania �wicze� laboratoryjnych), powinna zapozna� si� z instrukcj� zawart� w dodatku A. Materia�y zamieszczone w dodatku B maj� u�atwi� rozw�j i modyfikacje projektu.


\include{opisy/opis_srodowiska}
\section{Okienka}
\subsection{Konfiguracja �wiata}
\begin{figure}[htbp]
 \centering
 \includegraphics[width=1.0\textwidth]{opisy/materialy/konfiguracja_swiata.jpeg}
 \label{fig:konfiguracja_swiata}
 \caption{Okno konfiguracji �wiata}
\end{figure}
\subsection{Zarz�dzanie robotami}
Okienko \texttt{Zarz�dzanie robotami} umo�liwia ustawianie pozycji oraz orientacji wybranego robota na scenie. Zak�adki umo�liwiaj� wyb�r \textit{Pioneera}, kt�rego pozycj� chcemy zmieni�. Aktywne s� tylko zak�adki skojarzone z~robotami dodanymi aktualnie do �wiata. W~polach odpowiedzialnych za pozycj� i~orientacj� robota kolor informuje o~poprawno�ci wprowadzonych danych -- zielony oznacza poprawnie wype�nione pola, natomiast czerwony b��dnie.

\begin{figure}[htbp]
 \centering
 \includegraphics[]{opisy/materialy/zarzadzanie_robotami.jpeg}
 \label{fig:zarzadzanie_robotami}
 \caption{Okno zarz�dzania robotami}
\end{figure}

\subsubsection{Implementacja}
Klasa odpowiadaj�ca za ca�e okienko to \texttt{RobotManagementWindow}. Dziedziczy ona po klasie \texttt{QDialog}. Jej kluczowe metody to:
\begin{lstlisting}[language=C++, numbers=none]
public slots:
	void onAddNewRobot(int id);
	void onHideRobot(int id);
\end{lstlisting}
Odpowiadaj� one za to co dzieje si� w~oknie odpowiednio w~momencie dodania robota i~schowania go. Mianowicie, w~momencie otrzymania odpowiednich sygna��w aktywowana b�d� dezaktywowana jest stosowna zak�adka.

Klasa odpowiadaj�ca za wygl�d zak�adki to \texttt{RobotManagementTab}. Dziedziczy ona po klasie \texttt{QWidget}. Jej kluczowe metody to:
\begin{lstlisting}[language=C++, numbers=none]
void receivedMsg(const boost::shared_ptr<const gazebo::msgs::Int> &msg);
private slots:
	void on_pushButtonUstaw_clicked();
	void on_pushButtonReset_clicked();
\end{lstlisting}














\subsection{Wyniki symulacji}

\subsubsection{Funkcjonalno��}

Okno wykres�w oferuje w pierwszej zak�adce podgl�d na pozycj� X-Y wszystkich robot�w dost�pnych na scenie. 

\includegraphics[height=10cm]{opis_systemu/wyniki_symulacji/images/robot_trace.pdf}

Na kolejnych wykresach ilustrowana jest pozycj� robota w osiach X,Y oraz orientacje wzgl�dem czasu symulacji.

\includegraphics[height=10cm]{opis_systemu/wyniki_symulacji/images/robot_pose_2.pdf}

Okno zawiera dwa przyciski pozwalaj�ce na zresetowanie danych o robotach (czas symulacji nie zostaje zresetowany). Oraz przycisk pozwalaj�cy na zapis danych.

\includegraphics[height=10cm]{opis_systemu/wyniki_symulacji/images/zrzut_2.png}

Dane znajduj� si� w katalogu g��wnym kontenera. Wykresy s� zapisywane w postaci plik�w pdf. Program generuje r�wnie� plik csv z danymi uzyskanymi podczas symulacji. \\ \#Pionier1 X,Pionier1 Y,Pionier1 W,Pionier2 X,Pionier2 Y,Pionier2 W,Pionier3 X,Pionier3 Y,Pionier3 W,time\\
2.4122,-1.15,1.6118,-0.63934,-0.040688,0.047668,0,1,1,1.389\\
2.4082,-1.1215,1.5783,-0.61012,-0.036784,0.01829,0,0,0,1.409\\
2.4072,-1.1129,1.5685,-0.60134,-0.035781,0.0094779,0,0,0,1.415
\subsubsection{Implementacja}

Do stworzenia wykres�w wykorzystano biblioteki QT i QCustomPlot. Klasa ResultWindow odpowiedzialnej za rejestrowanie, wy�wietlanie i zapisywanie informacji o robotach. W klasie z wykorzystaniem funkcji SubscriberPtr dost�pnej w bibliotece gazebo/transport/transport.hh pobierane s� z topic-�w informacje o pozycji robot�w.

\bibliographystyle{plabbrv}
\nocite{*}
\bibliography{bibliografia}

\end{document}
