\section{Pluginy}
Plugin, czyli wtyczka, jest cz�ci� kodu, kt�ry zosta� skompilowany jako wsp�dzielona biblioteka i dodany do symulatora. Ma on dost�p do wszystkich funkcjonalno�ci Gazebo poprzez gotowe klasy j�zyka C++. Pluginy s� bardzo przydatne nie tylko ze wzgl�du na mo�liwo�� kontroli dowolnych modu��w symulatora, ale r�wnie� ze wzgl�du na swoj� elastyczno��. Z �atwo�ci� mo�na je dodawa� i usuwa� z systemu. W Gazebo jest dost�pnych 6 typ�w wtyczek:
\begin{itemize}
\item World
\item Model
\item Sensor
\item System
\item Visual
\item GUI
\end{itemize}
Oprogramowanie utworzone w ramach projektu wykorzystuje dwie z nich: GUI i World.
\subsection{GUI}
Plugin GUI jest bezpo�rednio zwi�zany z graficznym interfejsem u�ytkownika. Z jego poziomu mo�na w prosty spos�b dodawa� elementy takie jak przyciski, okna, listy czy pola tekstowe.
\subsection{World}
World plugin pozwala na edycj� r�nych parametr�w �wiata, przyk�adowo silnika fizyki lub o�wietlenia. Ponadto umo�liwia modyfikowanie sceny i znajduj�cych si� na niej obiekt�w.