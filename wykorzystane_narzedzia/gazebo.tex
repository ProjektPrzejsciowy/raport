 \section{Gazebo}
  
 \begin{figure}[htbp]
  \centering
  \includegraphics[width=0.4\textwidth]{wykorzystane_narzedzia/materialy/gazebo.jpg}
  \label{fig:blender}
 \end{figure}
 
  Symulator robotów jest doskonałym narzędziem dla każdej osoby zajmującej się robotyką. Pozwala szybko przetestować różne algorytmy i konstrukcje oraz skomplikowane systemy realizujące niecodzienne scenariusze. Jednym z takich narzędzi jest darmowy program Gazebo, przeznaczony do tworzenia dokładnych i efektywnych symulacji 3D robotów działających w złożonych środowiskach. Posiada zaawansowany silnik fizyki, wysokiej jakości grafikę oraz wygodne i programowalne interfejsy. Symulator dostępny jest dla systemu Linux na licencji Apache 2.0.
  Program Gazebo powstał na Uniwersytecie Południowej Kalifornii jako część systemu The Player Project, w którego skład wchodził również symulator 2D - Stage. W 2011 roku symulator ten został zintegrowany z bibliotekami Robot Operating System(ROS), co przyczyniło się do znacznego zwiększenia jego popularności. Rozwojem Gazebo zajmuje się teraz Open Source Robotic Foundation. Silnik graficzny wykorzystany w Gazebo to OGRE (Object Oriented Graphics Rendering Engine). Odpowiada on za wygląd użytych modeli, oświetlenie oraz cienie. Poprawnie odwzorowaną fizykę (między innymi model zderzeń i kolizji) zapewnia silnik ODE (Open Dynamics Engine). Gazebo daje możliwość symulowania takich sensorów jak: czujniki odległości, kamery, skanery 3D czy moduły GPS. Program pozwala na dodanie do każdego modelu kontrolera, sterującego jego pracą.Możliwe jest tworzenie prostych modeli bezpośrednio w Gazebo oraz importowanie bardziej złożonych, zaprojektowanych w zewnętrznych programach do grafiki 3D.
  
Podczas wyboru symulatora najistotniejsza była możliwość integracji z ROSem, a pod tym względem Gazebo jest bezkonkurencyjne. Nie bez znaczenia jest również dobra dokumentacja dostępna na stronie http://gazebosim.org/.
