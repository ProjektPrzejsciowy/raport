\chapter{Instrukcja dla dewelopera}
\section{Wst�p}

W niniejszej instrukcji znajduje si� opis wst�pnych czynno�ci wymaganych do dalszego rozwoju symulatora. Prezentuje konfiguracj� narz�dzi, �rodowiska oraz obrazu a tak�e automatyzacj� procesu budowy obrazu dla wersji deweloperskiej oraz testowej/finalnej przy pomocy pliku Dockerfile.\\
Instrukcja opisuje stan na stycze� 2017 roku, przez co informacje w niej zawarte mog� by� nieaktualne. Dotyczy to system�w operacyjnych wspieraj�cych Dockera, dost�pnej wersji ROS'a oraz Gazebo, a tak�e API, z kt�rego korzystano.  
%
\section{System}
\label{sec:system}
Wymagany jest system Linux w architekturze 64 bitowej. Wybrane wspierane dystrybucje \textbf{Ubuntu}:
\begin{itemize}
	\item Yakkety 16.10
	\item Xenial 16.04 (LTS)
	\item Trusty 14.04 (LTS)
\end{itemize}
\textbf{Debian}:
\begin{itemize}
	\item Jessie 8.0 (LTS)
	\item Wheezy 7.7 (LTS)
\end{itemize}
Pe�na lista wspieranych dystrybucji dost�pna jest pod adresem \url{https://docs.docker.com/engine/installation/linux/}.
Praca z Dockerem mo�liwa jest r�wnie� na systemie Windows, ale niniejsza instrukcja nie zawiera opisu dla tego systemu, gdy� nie uda�o si� skonfigurowa� �rodowiska do pracy z Dockerem w trybie graficznym.

\section{Docker}
Oprogramowanie nale�y zainstalowa� zgodnie z instrukcj� przygotowan� dla wybranego systemu operacyjnego umieszczon� na stronie dostawcy (link z podpunktu \ref{sec:system}).

\section{Obraz}
Symulator mo�na rozbudowywa�, dzi�ki przygotowanemu obrazowi, dost�pnego na~platformie \textbf{DockerHub} pod adresem \url{https://hub.docker.com/r/projektprzejsciowy2016/docker_lab/}. Sk�ada si� z systemu operacyjnego \textbf{Ubuntu Xenial 16.04 LTS}, systemu \textbf{ROS Kinetic} oraz \textbf{Gazebo 7.5} oraz pozosta�ych narz�dzi do komunikacji ROS-Gazebo. Je�eli zaprezentowana konfiguracja jest akceptowalna przez dewelopera, nale�y pobra� obraz. W przeciwnym wypadku nale�y zapozna� si� z rozdzialem \ref{sec:dockerfile}, w kt�rym przedstawiono, jak przy pomocy pliku Dockerfile mo�na wygenerowa� nowy obraz.

\section{Dockerfile}
\label{sec:dockerfile}